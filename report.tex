\documentclass[a4paper,12pt]{article}

\usepackage{listings}
\usepackage{color}

\lstset{frame=tb,
	language=Java,
	basicstyle={\small\ttfamily},
	breaklines=true,
	breakatwhitespace=true,
	tabsize=2
}

\begin{document}

\begin{center}
{\Large Computer Vision Coursework 3} \\
{\Large Scene Recognition} \\
Lucas Noyau - ln3g14 - 26921936\\
Jean-Luc Mourey - jm32g15 - 28231589\\
\today
\end{center}


% ----------------------------------------------------------------------------
% ----------------------------------------------------------------------------
\section{Introduction}
This coursework required the development of three different classifiers. The first uses a simple k-nearest-neighbour algorithm. The second uses a set of linear classifiers. The third classifier was given no specifications, and it was left to us to create the best possible classifier.

% ----------------------------------------------------------------------------
% ----------------------------------------------------------------------------
\section{Classifiers}
All classifiers used in this project extend from the abstract class \texttt{MyClassifier}. It was created in order to limit the amount of repeated code in the other classifiers, as well as give a framework for us to work from. The content of the class is as follows:
\begin{itemize}
	\item Class variables are \texttt{trainingData} and \texttt{testingData}. These hold the datasets that are used for training and testing respectively.
	\item Constructors, one with no parameters, the other uses a String of the path to get the datasets from.
	\item \texttt{go()} method is used to train and test the classifier on the class variable datasets, with the result being saved by \texttt{printResults(ArrayList)}.
	\item \texttt{printResults(ArrayList)} takes a list of the predicted classes in String form, and both prints it to \texttt{System.out} and to a file \texttt{output.txt}.
	\item \texttt{classify(groupedDataset)} takes a dataset and returns the ArrayList of String that is used by \texttt{printResults}. It does this by iterating over all the images in the dataset and calling \texttt{classify(FImage)} on each of them.
	\item \texttt{train(GroupedDataset)} takes a dataset but doesn't return anything. Each classifier has a different method for training, therefore this method is abstract.
	\item \texttt{classify{FImage}} is another abstract method, as image classification depends on the classifier. This method return a String of the predicted class name.
\end{itemize}
Therefore the only variations in the classifiers listed below are in the \texttt{train(GroupedDataset)} and \texttt{classify(FImage)} methods, although other methods are used in order to remove duplicate code and make the code easier to read.

% ----------------------------------------------------------------------------
\subsection{Run 1: A Simple k-nearest-neighbour Classifier}
\begin{itemize}
	\item \texttt{Run1(String, String)}, the constructor. Calls the constructor of its parent class by passing \texttt{trainingDataPath} and \texttt{testingDataPath}.
	\item \texttt{void train(GroupedDataset<String,ListDataset<FImage>,FImage>)} generates the feature vectors of each image in the training dataset and saves them in a \texttt{DoubleNearestNeighboursExact} object. Calls the method below.
	\item \texttt{extractFeature(GroupedDataset<String, ListDataset<FImage>, FImage>)} iterates over every image in every class, calling \texttt{extractFeature(FImage)} on each. Returns the list of all the extracted feature vectors.
	\item \texttt{extractFeature(FImage)}, which crops the image to a square and resizes it to 16*16 pixels. Returns the new image's feature vector values.
	\item \texttt{@Override String classify(FImage)} which takes an image, gets its nearest neighbours, and returns the predicted class, which is generated by counting the classes of the neighbours to find the most likely class.
\end{itemize}

% ----------------------------------------------------------------------------
\subsection{Run 2: A Set of Linear Classifiers}
\begin{itemize}
	\item \texttt{Run2(String, String)}, the constructor. Calls the constructor of its parent class like in run 1.
	\item \texttt{train(GroupedDataset<String,ListDataset<FImage>,FImage>)} trains an assigner with a vocabulary created from the set of training images. It then generates the linear classifier from a generated feature extractor which is trained with the data.
	\item \texttt{trainQuantiser(Dataset<FImage>)} iterates through every image in the dataset and adds all the values from the feature vectors of each image's rows to a float array list (List<float[]>, not ArrayList<float>)
	\item \texttt{extractFeature(FImage)} scans across the image, generates feature vectors, and returns the feature list.
	\item \texttt{Extractor} a class that implements \texttt{FeatureExtractor}. uses bag of words to extract features from the given image. is called by the \texttt{train()} method.
	\item \texttt{classify(FImage)} is used to pass the image against the \texttt{LiblinearAnnotator.classify()} method, returning the predicted class as a String.
\end{itemize}

% ----------------------------------------------------------------------------
\subsection{Run 3: Developing The Best Possible Classifier}
\begin{itemize}
	\item \texttt{Run3(String, String)}, the constructor. Calls the constructor of its parent class like in run 1.
	\item \texttt{train(GroupedDataset<String,ListDataset<FImage>,FImage>)}
	\item \texttt{classify(FImage)}
\end{itemize}

% ----------------------------------------------------------------------------
% ----------------------------------------------------------------------------
\section{Individual Contributions}
All progress made in this coursework was done together in labs. While Lucas was at the helm when it comes to the coding, Jean-Luc researched gaps in our knowledge, like OpenIMAJ classes and methods as well as general information on computer vision concepts; while also helping with identifying issues in code and debugging.
\textbf{(I feel like you should rewrite what you contributed, because the way I wrote it almost makes it sounds like I did more than you lmfao)}

\end{document}