\documentclass[12pt]{article}

\usepackage{listings}

\lstset{frame=tb,
	language=Java,
	basicstyle={\small\ttfamily},
	breaklines=true,
	breakatwhitespace=true,
	tabsize=2
}

\begin{document}

\begin{center}
{\Large Computer Vision Coursework 3} \\
{\Large Scene Recognition} \\
Lucas Noyau - ln3g14 - 26921936\\
Jean-Luc Mourey - jm32g15 - 28231589\\
\today
\end{center}


% ----------------------------------------------------------------------------
% ----------------------------------------------------------------------------
\section{Introduction}

% ----------------------------------------------------------------------------
% ----------------------------------------------------------------------------
\section{Classifiers}
All classifiers used in this project extend from the abstract class \texttt{MyClassifier}. It was created in order to limit the amount of repeated code in the other classifiers, as well as give a framework for us to work from. The content of the class is as follows:
\begin{itemize}
	\item Class variables are \texttt{trainingData} and \texttt{testingData}. These hold the datasets that are used for training and testing respectively.
	\item Constructors, one with no parameters, the other uses a String of the path to get the datasets from.
	\item \texttt{go()} method is used to train and test the classifier on the class variable datasets, with the result being saved by \texttt{printResults(ArrayList)}.
	\item \texttt{printResults(ArrayList)} takes a list of the predicted classes in String form, and both prints it to \texttt{System.out} and to a file \texttt{output.txt}.
	\item \texttt{classify(groupedDataset)} takes a dataset and returns the ArrayList of String that is used by \texttt{printResults}. It does this by iterating over all the images in the dataset and calling \texttt{classify(FImage)} on each of them.
	\item \texttt{train(GroupedDataset)} takes a dataset but doesn't return anything. Each classifier has a different method for training, therefore this method is abstract.
	\item \texttt{classify{FImage}} is another abstract method, as image classification depends on the classifier. This method return a String of the predicted class name.
\end{itemize}
Therefore the only variations in the classifiers listed below are in the \texttt{train(GroupedDataset)} and \texttt{classify(FImage)} methods, although other methods are used in order to remove duplicate code and make the code easier to read.

% ----------------------------------------------------------------------------
\subsection{Run 1: A Simple k-nearest-neighbour Classifier}

% ----------------------------------------------------------------------------
\subsection{Run 2: A Set of Linear Classifiers}

% ----------------------------------------------------------------------------
\subsection{Run 3: Developing The Best Possible Classifier}


% ----------------------------------------------------------------------------
% ----------------------------------------------------------------------------
\section{Individual Contributions}


% ----------------------------------------------------------------------------
% ----------------------------------------------------------------------------
\appendix
\section{Code}

\end{document}